\documentclass{article}
\usepackage{amsmath,amssymb,latexsym, color}

\newcommand{\F}[1]{\operatorname{#1}}
\newcommand{\TEXT}{\mathbf{TEXT}}
\newcommand{\template}[2]{\mathtt{#1}\left<#2\right>}


\begin{document}
\title{Building Graph Indexes for Entity/Relational Databases}
\author{The Justice League}
\date{June, 2011}
\maketitle

\section{The entity relational data model}

\begin{tabular}{|l|p{4in}|} \hline
$\mathcal{E}$ & the collection of all entity sets. This corresponds to a
collection of tables. \\
$E$ & some specific entity set.  This corresponds to some specific table.\\
$e$ & some specific entity.  This corresponds to some specific tuple in a
table. \\\hline\hline
$\mathcal{R}$ & the collection of all relationships.  This can be a collection
of foreign key constraints, or SQL queries. \\
$R$ & some specific relationship.  A relation can be specified by either
foreign key constraints or SQL queries. \\
$r$ & some specific relationship entry.  This corresponds to a tuple from a
multi-way join query. We will distinguish $r$ and $R$ by referring to them as
entity groups and relationships.\\ \hline\hline
$\mathbf{A}$ & the universe of all possible attribute values.  This is the
collection of all column names in the relational database. \\
$\mathbf{U}$ & the universe of all possible tuple cell values.  This is really
for the convenience of formal definitions. \\
$\mathbf{U}(\alpha)$ & the universe of all possible values of a particular
attribute $\alpha\in\mathbf{A}$. \\ \hline
$\alpha$ & an attribute from $\mathbf{A}$.  Its domain is
$\mathbf{U}(\alpha)$.\\
$\beta$ & a link of some relationship. \\\hline
\end{tabular}

\subsection{Functions on entities and entity sets}
Given an entity $e$, it has a set of attributes:
$$\F{attr}(e)\subseteq\mathbf{A}$$
Formally, the entity is a function $e:\F{attr}(e)\to\mathbf{U}$.
Each entity has a unique identifier denoted by $\F{id}(e)$.

An entity set $E$ is a collection of entities with the same attributes:
$$\forall e, e'\in E,\ \F{attr}(e) = \F{attr}(e')$$
Each entity set has a name: $\F{name}(E)$.

Therefore, it makes sense to define $\F{attr}(E)$ to make the attributes of the
entity set $E$.

\subsection{Functions on entity groups and relationships}

It's important to distinguish the two concepts: a relationship is defined by
either foreign key constraints or SQL queries, while an entity group is a tuple
that belongs to the join query of a relationship.

Let $R$ be a relationship.  It's defined by links and attributes.
Links and attributes are subsets of the attribute names.
$$\F{link}(R)\subseteq\mathbf{A}$$
$$\F{attr}(R)\subseteq\mathbf{A}$$
The difference between links and attributes are that links refer to entity sets,
while attributes refer to additional data introduced by the relationship.

We can think of $R$ as a function (in the same spirit as entities) over the
links.
\begin{eqnarray*}
  R &:& \F{link}(R) \to \mathcal{E} \\
\end{eqnarray*}

Each entity group $r$ of $R$ is a function that maps links of $R$ to entities
from the respective entity sets, and attributes of $R$ to some attribute value.

Given a link $\beta\in\F{link}(R)$, we have 
$r(\beta)\in R(\beta)$, and for an attribute $\alpha$,
$r(\alpha)\in\mathbf{U}(\alpha)$.

\subsection{An example}

\textcolor{red}{Do this as a homework.}

\section{Graph Index}

\begin{tabular}{|l|p{4in}|}\hline
$d$ & a document \\
$\gamma$ & a field in a document \\
$\TEXT$ & all possible text values.\\
$\mathbf{I}$ & An index.\\ \hline
\end{tabular}

A relational database is completely characterized by:
$\mathbf{DB} = (\mathcal{E}, \mathcal{R})$.  In this section, we will specify
the index structures of the database.

\subsection{Documents: a universal element}

Text indexes are completely unstructured in the sense that it can only contain
documents.  A document $d$ is characterized by
\begin{itemize}
\item its attributes: $\F{field}(d)$
\item its field values.  We treat $d$ as a function of the type
$$d:\F{field}(d)\to\TEXT$$
So, given a field $\gamma$, the text value is given by $d(\gamma)$.
\end{itemize}

Note, different documents may have different fields.

The constructor of a field is denoted as $\gamma(x, y)$ where $x$ is the field
label, and $y$ the value.

\subsection{Indexing functions}

A (generic) text analyzer is a function:

$$\Phi:\TEXT \to \template{list}{\TEXT}$$
which maps a text string to a sequence of text strings.

An field specific analyzer is an analyzer that depends on the field:
$$\Phi : \F{field}(d)\times\TEXT \to\template{list}\TEXT$$
So field specific analyzers takes two arguments $(\gamma\in\F{field}(d),
x\in\TEXT)$.  We may denote $\Phi_\gamma = \lambda x.\ \Phi(\gamma,x)$.

\subsection{A text index}
A text index $\mathbf{I} = \left<\mathcal{D}, \Phi\right>$ is a collection of
documents $\mathcal{D} = \{d_1, d_2, \dots, d_n\}$, and analyzer (generic or
field specific) $\Phi$.

The objective of our research is to investigate the following problems:
\begin{itemize}
\item mapping between relational databases $\mathbf{DB}$ and text indexes
$\mathbf{I}$.
\item answer complex queries concerning $\mathbf{DB}$ using the corresponding
text index $\mathbf{I}$.
\item map the query answers from the text index $\mathbf{I}$ back to the
relational data model of $\mathbf{DB}$.
\end{itemize}

\textcolor{red}{We actually should be very precise at the {\em types} of queries
that we wish to support even if we intend to deal with keyword queries.}

\section{Mapping relational data model to document data model}

Given a database $\mathbf{DB} = (\mathcal{E}, \mathcal{R})$.  We wish to define
a mapping $h:\mathbf{DB}\mapsto \mathbf{I}$.

\subsection{Mapping entities to documents}

Given an entity $e\in E$, we need to define the document $h(e)$.  Each attribute
$\alpha\in\F{attr}(e)$, and its value $e(\alpha)$ is mapped to a field in the
document.

The attribute-based fields are defined as:
$$A(e) = \{\gamma(\alpha, e(\alpha)) : \alpha\in\F{attr}(e)\}
 \cup \left\{\gamma\left(*,
\sum_{\alpha\in\F{attr}(e)}e(\alpha)\right)\right\}$$

The schema-based fields are defined as:
$$B(e) = \{\gamma(\F{ent-set}, \F{name}(E)),
           \gamma(\F{ent-id}, \F{id}(e))\}$$

We introduce a special field $*$ that consists of the concatenation of the
values of all the attributes of $e$.

The document $h(e)$ is defined as 
$\F{doc}(A\cup B)$.

\textcolor{red}{We are not indexing the attribute names by $h$.  This is not
good because we cannot query using ``{\em course title database}''.  An {\em
easy} way of dealing with it is to add $\F{name}(\alpha)$ to $e(\alpha)$ in the
fields.}

\subsection{Mapping relationships to documents}

Let $R$ be a relationship, and $r\in R$ be an entity group.  We define three
groups of fields in the document $h(r)$.

\begin{eqnarray*}
L(r) &=& \{\gamma(\beta,\F{id}(r(\beta))) : \beta\in\F{links}(r)\} \cup
     \left\{
        \gamma\left(**,\sum_{\beta\in\F{links}(r)}\F{id}(r(\beta))\right)
      \right\} \\
A(r) &=& \makebox{ as defined above for entities} \\
B(r) &=& \{\gamma(\F{relationship}, \F{name}(R))\}
\end{eqnarray*}

Finally, $h(r) = \F{doc}(A(r)\cup B(r)\cup L(r))$.

The resulting document $h(r)$ has two {\em default} fields, $*$ and $**$.  The
former represents all text values of its data attributes, and the latter
represents all entities that it connects to.

\subsection{Functions on the index}

Collectively, we refer to the resulting documents $\{h(e)\}\cup\{h(r)\}$ as the
text index $\textbf{I}$ of the database $\textbf{DB}$.

Recall, $\textbf{I}$ requires the specification of an analyzer $\Phi$ as well.
If $\Phi$ is a simple generic analyzer, then the text can support keyword
search, but if the more space costly $q$-gram analyzer is used, then the index
can also support approximate string matching.  We note that we can use field
specific analyzers $\Phi_\gamma$ so that certain fields are indexed using simple
analyzers and others are indexed using $q$-grams.  \textcolor{red}{But this
causes interesting headaches when creating a query...}

The search function is denoted by a function
$\mathbf{I}:\TEXT\to\template{list}{\mathcal{D}}$ that maps strings to a list of
documents.

\section{Search and query processing}

The text index $\mathbf{I} = h(\mathbf{DB})$ assists in the query answering
because it is capable of keyword searches and approximate string matching.

\subsection{Queries}
\label{sec:query}
A query is an operation that retrieves a {\em set} of entities or entity-groups.
We define a query using a formal grammar.
\begin{eqnarray*}
\mathrm{query} 
  &\rightarrow& (\mathtt{:is}\ \mathrm{doctype}) \\
  &|& (\mathtt{:in}\ \mathrm{table}) \\
  &|& (\mathtt{:has}\ \mathrm{field}) \\
  &|& (\mathtt{:contains}\ \mathrm{id}) \\
  &|& (\mathtt{:contains}\ \mathrm{id}\ \mathtt{:as}\ \mathrm{link}) \\
  &|& (\mathtt{:keyword}\ \mathrm{text})\\
  &|& (\mathtt{:keyword}\ \mathrm{text}\ \mathtt{:as}\ \mathrm{attribute})\\
  &|& (\mathtt{:and}\ \mathrm{query}+) \\
  &|& (\mathtt{:or}\ \mathrm{query}+) \\
  &|& (\mathtt{:not}\ \mathrm{query})\\
\mathrm{doctype}
  &\rightarrow& \mathtt{:entity}\ |\ \mathtt{:group} \\
\mathrm{table} &\rightarrow& \mathcal{E}\cup\mathcal{R} \\
\mathrm{field} &\rightarrow& \mathbf{A} \\
\mathrm{link} &\rightarrow& \mathbf{A} \\
\mathrm{attribute} &\rightarrow& \mathbf{A} \\
\mathrm{id} &\rightarrow& \{\mathrm{id}(e) : e\in\mathbf{E}\} \\
\mathrm{text} &\rightarrow& \TEXT \\
\end{eqnarray*}

Here is an example of a query that gets all entities containing the word
``Superman'' some where.
$$q = 
  (\mathtt{:and}\ (\mathtt{:is}\ \mathtt{entity})
    \ (\mathtt{:keyword}\ \textsf{``Superman''})) $$
If we want to get look up ``Superman'' under an attribute ``alias'', we need
something like:
$$q = 
  (\mathtt{:and}\ (\mathtt{:is}\ \textit{entity})
    \ (\mathtt{:keyword}\ \textsf{``Superman''}\ 
        \mathtt{:as}\ \textit{alias})) $$

Suppose that, through the query $q$, we have discovered that \textit{Superman}
is entity $P4012$, we can use the entity ID $P4012$ to find related entities:
$$q = 
  (\mathtt{:and}
   \ (\mathtt{:is}\ \textit{group})
   \ (\mathtt{:contains}\ P4012)) $$

If we are particularly interested at {\em Superman} as a superhero, we may focus
on groups which views $P4012$ as a hero:
$$q = 
  (\mathtt{:and}
   \ (\mathtt{:is}\ \textit{group})
   \ (\mathtt{:contains}\ P4012\ \mathtt{:as}\ \textit{hero})) $$

\subsection{Denotational Semantics}
The term {\em denotational semantics} refers to the formal definition of the
{\em correct} answer to a query in terms of the sets of elements it should
contain in its result set.  The denotational semantics of the query constructs
defined in Section~\ref{sec:query} is quite straight-forward.

\textcolor{red}{To be completed.}


\end{document}
