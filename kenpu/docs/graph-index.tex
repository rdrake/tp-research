\documentclass{article}
\usepackage{amsmath,amssymb,latexsym, color}

\newcommand{\F}[1]{\operatorname{#1}}
\newcommand{\TEXT}{\mathbf{TEXT}}
\newcommand{\template}[2]{\mathtt{#1}\left<#2\right>}


\begin{document}
\title{Building Graph Indexes for Entity/Relational Databases}
\author{The Justice League}
\date{June, 2011}
\maketitle

\section{The entity relational data model}

\begin{tabular}{|l|p{4in}|} \hline
$\mathcal{E}$ & the collection of all entity sets. This corresponds to a
collection of tables. \\
$E$ & some specific entity set.  This corresponds to some specific table.\\
$e$ & some specific entity.  This corresponds to some specific tuple in a
table. \\\hline\hline
$\mathcal{R}$ & the collection of all relationships.  This can be a collection
of foreign key constraints, or SQL queries. \\
$R$ & some specific relationship.  A relation can be specified by either
foreign key constraints or SQL queries. \\
$r$ & some specific relationship entry.  This corresponds to a tuple from a
multi-way join query. We will distinguish $r$ and $R$ by referring to them as
entity groups and relationships.\\ \hline\hline
$\mathbf{A}$ & the universe of all possible attribute values.  This is the
collection of all column names in the relational database. \\
$\mathbf{U}$ & the universe of all possible tuple cell values.  This is really
for the convenience of formal definitions. \\
$\mathbf{U}(\alpha)$ & the universe of all possible values of a particular
attribute $\alpha\in\mathbf{A}$. \\ \hline
$\alpha$ & an attribute from $\mathbf{A}$.  Its domain is
$\mathbf{U}(\alpha)$.\\
$\beta$ & a link of some relationship. \\\hline
\end{tabular}

\subsection{Functions on entities and entity sets}
Given an entity $e$, it has a set of attributes:
$$\F{attr}(e)\subseteq\mathbf{A}$$
Formally, the entity is a function $e:\F{attr}(e)\to\mathbf{U}$.
Each entity has a unique identifier denoted by $\F{id}(e)$.

An entity set $E$ is a collection of entities with the same attributes:
$$\forall e, e'\in E,\ \F{attr}(e) = \F{attr}(e')$$
Each entity set has a name: $\F{name}(E)$.

Therefore, it makes sense to define $\F{attr}(E)$ to make the attributes of the
entity set $E$.

\subsection{Functions on entity groups and relationships}

It's important to distinguish the two concepts: a relationship is defined by
either foreign key constraints or SQL queries, while an entity group is a tuple
that belongs to the join query of a relationship.

Let $R$ be a relationship.  It's defined by links and attributes.
Links and attributes are subsets of the attribute names.
$$\F{link}(R)\subseteq\mathbf{A}$$
$$\F{attr}(R)\subseteq\mathbf{A}$$
The difference between links and attributes are that links refer to entity sets,
while attributes refer to additional data introduced by the relationship.

We can think of $R$ as a function (in the same spirit as entities) over the
links.
\begin{eqnarray*}
  R &:& \F{link}(R) \to \mathcal{E} \\
\end{eqnarray*}

Each entity group $r$ of $R$ is a function that maps links of $R$ to entities
from the respective entity sets, and attributes of $R$ to some attribute value.

Given a link $\beta\in\F{link}(R)$, we have 
$r(\beta)\in R(\beta)$, and for an attribute $\alpha$,
$r(\alpha)\in\mathbf{U}(\alpha)$.

\subsection{An example}

\textcolor{red}{Do this as a homework.}

\section{Graph Index}

\begin{tabular}{|l|p{4in}|}\hline
$d$ & a document \\
$\gamma$ & a field in a document \\
$\TEXT$ & all possible text values.\\
$\mathbf{I}$ & An index.\\ \hline
\end{tabular}

A relational database is completely characterized by:
$\mathbf{DB} = (\mathcal{E}, \mathcal{R})$.  In this section, we will specify
the index structures of the database.

\subsection{Documents: a universal element}

Text indexes are completely unstructured in the sense that it can only contain
documents.  A document $d$ is characterized by
\begin{itemize}
\item its attributes: $\F{field}(d)$
\item its field values.  We treat $d$ as a function of the type
$$d:\F{field}\to\TEXT$$
So, given a field $\gamma$, the text value is given by $d(\gamma)$.
\end{itemize}

Note, different documents may have different fields.

\subsection{Indexing functions}

A text analyzer is a function:

$$\Phi:\TEXT \to \template{list}{\TEXT}$$
which maps a text string to a sequence of text strings.



\end{document}
